\documentclass[a4paper,12pt]{article}
\usepackage[margin=2.5cm]{geometry}

\usepackage{amsmath}
\usepackage{amssymb}
\usepackage{cancel}
%\usepackage{gensymb}
\usepackage{graphicx}
\usepackage{esint}
\usepackage{mdsymbol}
\usepackage{esvect} 
%\usepackage{lipsum}
\usepackage{hhline}
\usepackage{eurosym}
\usepackage{listings}
\usepackage{booktabs}
\usepackage{amssymb}
\usepackage{mathrsfs}
\usepackage{commath}
\usepackage{adjustbox}
\usepackage{booktabs}
\usepackage{array}
\usepackage{lscape}
\usepackage[xspace]{ellipsis}
\usepackage{color}
\usepackage{float}
\usepackage{caption}
\usepackage{afterpage}
\makeatletter
\lst@InstallKeywords k{attributes}{attributestyle}\slshape{attributestyle}{}ld
\makeatother
\lstset{language=C,breaklines=true,
	basicstyle=\fontsize{11}{13}\selectfont\ttfamily,
	keywordstyle=\color{blue}\ttfamily,
	stringstyle=\color{red}\ttfamily,
	commentstyle=\color{green}\ttfamily,
	morecomment=[l][\color{magenta}]{\#},
	morekeywords={uint8\_t, uint16\_t, uint32\_t, interrupt},
	moreattributes={ISPI, SPIDAT, CS, T0\_OVERFLOW\_MAX, TH1, TH0, TL1, TL0, EDGE_VALUE}, % etc...
	attributestyle = \bfseries\color{mymauve} % (for instance)
}

\definecolor{mygreen}{rgb}{0,0.6,0}
\definecolor{mygray}{rgb}{0.5,0.5,0.5}
\definecolor{mymauve}{rgb}{0.58,0,0.82}
\begin{document}
\title{7-Segment SPI Slave Display}
\author{Conor Dooley}
\maketitle
\section*{Display Behaviour}
The display has 10 8-bit registers to control which digits appear on the screen. Register 0 acts as a bitmask to control which digits of the 7-Segment display are enabled. Registers 1 through 8 contain the value to display in the corresponding location, with index 1 appearing at the right hand side of the board and index 8 on the far left. Register 9 will control the radix point. These registers can be written to via SPI and at startup all will contain zero.\\
Each bit in Register 0 and 9 are active high. Writing all zeros to these registers will disable all digits or radices respectively.
Registers 1 through 8 expect an 8 bit unsigned integer of which only the lower 4 bits will be used. The value will be automatically decoded into the appropriate segments.
\section*{SPI Behaviour}
%TODO maximum permissable frequency
The display can be controlled over SPI with the signals \texttt{spi\_sclk\_i}, \texttt{spi\_ss\_i} and \texttt{spi\_mosi\_i}, \texttt{spi\_miso\_o} all of which are idle high. The data is clocked on the positive edge of \texttt{spi\_sclk\_i}. The maximum transfer frequency is not yet established but the display is clocked at 5 MHz so the recommended maximum SPI clock is 25 MHz.\\
Each transaction is 16 bits long, which is transferred in two 8 bit sections. The upper byte is sent first and each byte is MSB first. The 16 bit long message is broken up as follows:
\begin{table}[!h]
	\begin{center} 
		\begin{tabular}{|c|c|c|c| c|c|c|c| c|c|c|c| c|c|c|c|}
			15 & 14 & 13 & 12 & 11 & 10 & 09 & 08 & 07 & 06 & 05 & 04 & 03 & 02 & 01 & 00 \\
			\multicolumn{4}{|c|}{command} & \multicolumn{4}{c|}{address} & \multicolumn{8}{c|}{display value}			
		\end{tabular}
	\end{center}
\end{table}
The only command that is currently valid is the write command of \texttt{0001}. The address is the 4 bit unsigned integer representation of the register number.


	
\end{document}